\documentclass[a4paper]{article}
\usepackage{amsmath}
\usepackage{graphicx}
\usepackage{geometry}
\usepackage{floatrow}
\usepackage{layout}
\usepackage{amssymb} 
\usepackage{multirow}
\usepackage{caption}
\geometry{margin=1in}
\usepackage{authblk}
\usepackage{indentfirst}
\usepackage[hidelinks]{hyperref}
\usepackage{enumitem}
\usepackage{float}

\providecommand{\keywords}[1]
{
  \small	
  \textbf{\; \textit{Keywords---}} #1
}

\begin{document}

\title{\textbf{\huge{Problem Set 1}}}

\author{\textbf\large{Teera Tesharojanasup}}

\affil{\textbf{Northeastern University, Boston}}

\date{\text{July 10th, 2024}}

\maketitle
\begin{sloppypar}

\section*{Overview}

Problem set 1 for CS 4100 Summer II, year 2024.

\section{Categorizing AI Environments}

\begin{enumerate}[start=1,label=Q\arabic*,left=0pt]
    \item \textbf{Under what (implementation-related) assumptions is maze-solving a fully observable environment? What are the agent’s percepts?}
    \par We would assume that the agent is able to have all the information regarding the maze environment to be able to determine if the maze
    is fully observable. 
    This would include the complete maze layout like the ways it can traverse, the walls, the starting position, and the exit position.
    Therefore, the agent's percept would be:
    \begin{itemize}
        \item The complete map of the maze.
        \item The agent's current location.
        \item The goal/exit location.
        \item The possible turns/moves it could make from its current position.
    \end{itemize}
    
    \item \textbf{Under what (implementation-related) assumptions is maze-solving a partially observable environment? What are the agent’s percepts?}
    \par A maze is partially observable if the agent has limited information. This would mean that the agent
    will not have all the information regarding the maze like all the walls and the ways it can traverse. Therefore,
    the agent's percept would be:
    \begin{itemize}
        \item An incomplete map of the maze (created when the agent moves around the maze).
        \item The agent's current location.
        \item The agent may only have limited information about the goal location like its general direction but not its exact coordinate.
        \item The agent can only sense/see the maze in its immediate surrounding.
    \end{itemize}
    
    \item \textbf{Is a known environment always fully observable? Explain with an example.}
    \par No, consider the game of solitaire. The agent knows the rules of this environment where in this case would be the rules
    of the card game making it a known environment. However, it is only partially observable since the agent will not know which card will
    appear next because this information is hidden.
    
    \item \textbf{Is a partially observable environment always an unknown environment? Explain with an example.}
    \par No, consider an agent that delivers coffee within Google's headquarter. The agent has general information about the office building,
    like the number of floors, elevator locations etc. This environment is only partially observable because the agent has limited visibility due to
    its visual, and auditory sensors. 
    
    However, this is not an unknown environment since the agent has an understanding of the building layout and
    can navigate its way to a location following certain sets of rules such as not bumping into employees, moving on certain designated office highways,
    utilizing elevators to move up/down floors etc.

    \item \textbf{What is the difference between an unknown environment and a non-deterministic environment? Explain with an example.}
    \par An unknown environment is one where we do not know the rules. In a non-deterministic environment, there is the factor of randomness/uncertainty. A good example
    of this is to visualize a drone visiting an unexplored alien planet with the goal of finding life. This is an unknown environment since we do not have any prior maps or information 
    about this planet. 
    
    It is also non-deterministic because the chances of discovering life is not certain. The creation of life is complicated,
    and although we do know what creates life, we cannot say for certainty that just because building blocks of life exists, that life will exist.

\end{enumerate}

\section{Search}

\begin{enumerate}[start=6,label=Q\arabic*,left=0pt]
    \item \textbf{Can both Breadth-First Search (BFS) and Depth-First Search (DFS) be used to solve RJMs? Will both yield the shortest path?}
    \par Yes, BFS and DFS can be used to solve RJMs. This is because we can turn a RJM into an unweighted graph which BFS and DFS 
    can be used for. 
    
    Breadth-First Search will find the shortest path because it searches all the surrounding nodes in the same level. Thus, BFS
    will be able to go down multiple branches and compare them allowing us to find the shortest path. However, DFS may not 
    always find the shortest path depending on which branch DFS chooses.

    \item \textbf{Assuming that each jump has a cost of 1, irrespective of the number of cells the rook
    passes while making that jump, we can use A* search to find the shortest path. Recall
    from class that A* search requires a consistent heuristic to guarantee an optimal solution.
    Show that for this problem, the Manhattan distance from any cell to the goal cell divided
    by (n - 1) is a consistent heuristic, where n is the number of cells in a row/column. In other words, show that}
    \[
    H(node, goal, n) = \frac{|node_x - goal_x| + |node_y - goal_y|}{n - 1}
    \]
    \textbf{is a consistent heuristic. Remember that a proof may not rely on a single example, but instead must hold true in general.}
    
    \item \textbf{Consider the following RJM:}
    \begin{figure}[H]
        \centering  
        \includegraphics[height=0.2\textheight]{Q8_RJM.png}
        \label{fig:Q8_RJM}
    \end{figure}
    \textbf{On this maze, using the heuristic from Q7, complete the A* search process shown be- low, starting with Step 2. Terminate your search when the Goal node (cell D4) is popped from the priority queue. Double check your calculations!}

\end{enumerate}

\section{Local Search}

\begin{enumerate}[start=9,label=Q\arabic*,left=0pt]
    \item \textbf{Based on our discussions in class, is this problem well-suited to local search algorithms? Explain your reasoning.}
    \item \textbf{How you would implement such a program? What would be a reasonable objective function? (This question is somewhat open-ended, so don’t worry too much about the ‘right’ answer. I am really looking for whether your thought process reflects a solid understanding of how local search works.)}
    \item \textbf{Which of the optimizations and local search variants discussed in class (random restarts, varying step size, simulated annealing and genetic algorithms) are applicable to this problem, and why?}

\end{enumerate}

\section{Adversarial Search}

\begin{enumerate}[start=12,label=Q\arabic*,left=0pt]
    \item \textbf{Construct a game tree example with a depth of 4 and a branching factor of 2, that illustrates the best-case for alpha-beta pruning, except for the trivial solution. Hand drawn figures accepted for this question, but please ensure clarity.}
    \item \textbf{Why is it desirable to combine iterative deepening with alpha-beta pruning from a practical standpoint?}

\end{enumerate}

\section{Academic Integrity}

\begin{enumerate}[start=14,label=Q\arabic*,left=0pt]
    \item \textbf{Review, and copy/paste the academic integrity acknowledgement in your final submission as the answer to Q14.}
    \par I have read and understood the academic integrity policy as outlined in the course syllabus for CS4100. 
    By pasting this acknowledgement in my submission, I declare that all work presented here is my own, 
    and any conceptual discussions I may have had with classmates have been fully disclosed. I declare that generative AI 
    was not used to answer any questions in this assignment. Any use of generative AI to improve writing clarity 
    alone is accompanied by an appendix with my original, unedited answers.
\end{enumerate}
\end{sloppypar}

\cite{BOOK:1}
\cite{ARTICLE:1}
\bibliography{references}
\bibliographystyle{ieeetr}

\end{document}